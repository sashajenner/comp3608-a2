\section{Data}
\label{sec:data}

% This section should describe the dataset, mentioning the number of attributes and classes. It should also
% briefly describe the CFS method and list the attributes selected by the CFS.

The dataset used throughout this paper originates from the National Institute of Diabetes and Digestive and Kidney Diseases and was first used in a demonstration of the ADAP Learning Algorithm in 1988 \cite{pima}. It consists of 768 non-diabetic females aged at least 21 years old and of Pima Indian heritage. There are 9 columns per row, the first 8 of which are biometric measurement attributes whilst the final one is the class consisting of whether or not the individual with be diagnosed with diabetes. A description of each column in the dataset is shown in Table \ref{tab:cols}. To maintain consistency the dataset has been cleaned to remove any missing values.

\begin{table}[h!]
    \caption{A synopsis of the dataset's columns with those selected by CFS highlighted.\label{tab:cols}}
    \begin{center}
        \begin{tabular}{|m{14cm}|l|}
        \hline
        \textbf{Description} & \textbf{Units} \\
        \hline
        Number of times pregnant & n/a\nomenclature{n/a}{Not applicable} \\
        \rowcolor{lightblue}Plasma glucose concentration at 2 hours in an oral glucose tolerance test & mg/dL\nomenclature{mg/dL}{Milligrams per decilitre} \\
        Diastolic blood pressure & mm Hg\nomenclature{mm Hg}{Millimetres of mercury} \\
        Triceps skin fold thickness & mm\nomenclature{mm}{Millimetres} \\
        \rowcolor{lightblue}Serum insulin level & $\mu$U/mL\nomenclature{$\mu$U/mL}{Micro enzyme units per millilitre} \\
        \rowcolor{lightblue}Body mass index (BMI) & kg/m$^2$\nomenclature{kg/m$^2$}{Weight in kilograms per height in metres squared} \\
        \rowcolor{lightblue}Diabetes pedigree function (likelihood of diabetes based on family history) & n/a \\
        \rowcolor{lightblue}Age & years \\
        Is diabetes diagnosed between 1 and 5 years after the above measurements are recorded? & n/a\\
        \hline
    \end{tabular}
    \end{center}

    \subsection{Attribute Selection}
    The Correlation-based Feature Selection (CFS)\nomenclature{CFS}{Correlation-based feature selection} method is a way of determining a representative set of attributes which are highly correlated with the class but uncorrelated with each other. This can improve the training of a classification model by removing features that are not predictive of the class. \\

    Using the CFS algorithm \cite{cfs} implemented in Weka 3.8.5 \cite{weka}, the attributes that were selected are plasma glucose concentration, serum insulin level, BMI, diabetes pedigree function and age, and are additionally highlighted in Table \ref{tab:cols}.
\end{table}
